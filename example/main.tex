\documentclass[10pt]{national_olymp}
% \usepackage[utf8]{inputenc}

\input{misc/siunitx.cfg}

\title{qazcho_template}
\author{}
\date{}

% Когда нужен документ на казахском, нижняя команда не должна быть закомментирована
% \addto\captionsrussian{
%     \renewcommand{\contentsname}{Мазмұны}
%     \renewcommand{\figurename}{Сурет}
%     \renewcommand{\tablename}{Кесте}
% }

% Если латех не доперает, как разделить какое-то слово на слоги, можно ему помочь. Тогда не будет висюлек некрасивых. (Хотя это не всегда помогает, тогда придется вручную писать вставлять дефис, где надо)
\usepackage{hyphenat}
\hyphenation{вос-ста-но-ви-тель-но еті-ле-ді кристалло-гидрата}



\begin{document}

\SetLanguage{Russian}
% \SetLanguage{Kazakh}
\UsingTemplate{Tasks}
% \UsingTemplate{Solutions} % "Tasks" or "Solutions" or "Blank"

% \Practical

\titlepage{Районный}{2023-2024}{10}{chem10-1-blank}
% \titlepage{Аудандық}{2022-2023}{10}{chem10-1-tasks}

% \tableofcontents
% \clearpage

% \phantomsection
\section*{\fontspec{PT Sans}Олимпиада ережелері:}\vspace{5pt}
\addcontentsline{toc}{section}{Олимпиада ережелері}

\begin{regulations}
Сізге химия пәнінен 2022 жылғы республикалық олимпиаданың есептер жинағы берілді. Төмендегі нұсқаулар мен ережелердің барлығын \em{мұқият} оқып шығыңыз. Олимпиада тапсырмаларын орындау үшін сізде \textbf{5 астрономиялық сағат (300 минут)} беріледі. Сіздің жалпы нәтижеңіз - тапсырмалардың ұпай санын ескере отырып, әрбір тапсырма бойынша ұпайлар сомасы болып табылады.

Сіз шимайпарақта есептерді шеше аласыз, бірақ барлық шешімдерді жауап парақтарына көшіруді ұмытпаңыз. \textbf{Арнайы белгіленген жолақтардың ішіне жазған шешімдер ғана тексеріледі}. Шимайпарақтар \textbf{тексерілмейді}. Шешімдерді жауап парақтарына көшіру үшін сізге \textbf{қосымша уақыт берілмейтінін} ескеріңіз.

Сізге графикалық немесе инженерлік калькуляторды пайдалануға \textbf{рұқсат етіледі}.

Сізге кез келген анықтамалық материалдарды, оқулықтарды немесе жазбаларды пайдалануға \textbf{тыйым салынады}.

Сізге ішкі жадты немесе интернеттен жүктеп алынған мәтіндік, графикалық және аудио пішімінде ақпаратты сақтауға қабілетті кез келген байланыс құрылғыларын, смартфондарды, смарт сағаттарды немесе кез келген басқа гаджеттерді пайдалануға \textbf{тыйым салынады}.

Осы тапсырмалар жинағына кірмейтін кез келген материалдарды, соның ішінде периодтық кесте мен ерігіштік кестесін \textbf{пайдалануға рұқсат етілмейді}. \textbf{Мұқаба бетінде} периодтық жүйенің нұсқасы беріледі.

Кезең соңына дейін олимпиаданың басқа қатысушыларымен сөйлесуге \textbf{рұқсат етілмейді}. Ешбір материалдарды, соның ішінде кеңсе керек-жарақтарын өзара алмаспаңыз. Кез келген ақпаратты жеткізу үшін ымдау тілін қолданбаңыз.

Осы ережелердің кез келгенін бұзғаныңыз үшін сіздің жұмысыңыз \textbf{автоматты түрде 0 ұпаймен} бағаланады және бақылаушылар сізді аудиториядан шығаруға құқылы.

Жауап парақтарыңызға шешімдерді \textbf{анық} әрі \textbf{түсінікті} етіп жазыңыз. Қорытынды жауаптарды қарындашпен дөңгелектеу ұсынылады. \textbf{Өлшем бірліктерін көрсетуді ұмытпаңыз (өлшем бірліктері жазылмаған жауап есептелмейді)}. Арифметикалық амалдарда сандық мәліметтерді қолдану ережелерін сақтаңыз. Басқаша айтқанда, маңызды сандар бар екені есіңізде болсын.

Сәйкес есептерді бермей шешімнің соңғы нәтижесін ғана көрсетсеңіз, онда жауап дұрыс болса да \textbf{0} ұпай аласыз.

Бұл олимпиаданың шешімдері \href{https://www.qazcho.kz}{www.qazcho.kz} сайтында жарияланады.

Химия пәнінен олимпиадаға дайындық бойынша ұсыныстар \href{https://www.qazolymp.kz}{www.qazolymp.kz} сайтында берілген.
\end{regulations}
\phantomsection
\section*{\fontspec{PT Sans}Регламент олимпиады:}\vspace{5pt}
\addcontentsline{toc}{section}{Регламент олимпиады}

\begin{regulations}
Перед вами находится комплект задач республиканской олимпиады 2022 года по химии. \textbf{Внимательно} ознакомьтесь со всеми нижеперечисленными инструкциями и правилами. У вас есть \textbf{5 астрономических часов (300 минут)} на выполнение заданий олимпиады. Ваш результат – сумма баллов за каждую задачу, с учетом весов каждой из задач.

Вы можете решать задачи в черновике, однако, не забудьте перенести все решения на листы ответов. Проверяться будет \textbf{только то, что вы напишете внутри специально обозначенных квадратиков}. Черновики проверяться \textbf{не будут}. Учтите, что вам \textbf{не будет выделено} дополнительное время на перенос решений на бланки ответов.

Вам \textbf{разрешается} использовать графический или инженерный калькулятор.

Вам \textbf{запрещается} пользоваться любыми справочными материалами, учебниками или конспектами.

Вам \textbf{запрещается} пользоваться любыми устройствами связи, смартфонами, смарт-часами или любыми другими гаджетами, способными предоставлять информацию в текстовом, графическом и/или аудио формате, из внутренней памяти или загруженную с интернета.

Вам \textbf{запрещается} пользоваться любыми материалами, не входящими в данный комплект задач, в том числе периодической таблицей и таблицей растворимости. На \textbf{титульной странице} предоставляем единую версию периодической таблицы.

Вам \textbf{запрещается} общаться с другими участниками олимпиады до конца тура. Не передавайте никакие материалы, в том числе канцелярские товары. Не используйте язык жестов для передачи какой-либо информации.

За нарушение любого из данных правил ваша работа будет \textbf{автоматически} оценена в \textbf{0 баллов}, а прокторы получат право вывести вас из аудитории.

На листах ответов пишите \textbf{четко} и \textbf{разборчиво}. Рекомендуется обвести финальные ответы карандашом. \textbf{Не забудьте указать единицы} измерения \textbf{(ответ без единиц измерения будет не засчитан)}. Соблюдайте правила использования числовых данных в арифметических операциях. Иными словами, помните про существование значащих цифр.

Если вы укажете только конечный результат решения без приведения соответствующих вычислений, то Вы получите \textbf{0} баллов, даже если ответ правильный.

Решения этой олимпиады будут опубликованы на сайте \href{https://qazcho.kz}{www.qazcho.kz}.

Рекомендации по подготовке к олимпиадам по химии есть на сайте \\ \href{https://kazolymp.kz}{www.kazolymp.kz}.
\end{regulations}

\problem{Название крутой задачи}{11} % Название и вес задачи

\ProblemPointsFour{1}{2}{3}{3}{9}{11}

\begin{enumerate}
    \item Нарисуйте структуру вещества \CH{C2H6O} если известно, что в нем присутствует OH-группа.
    \item При какой температуре атомы гелия будут иметь среднеквадратичную скорость \qty{3.5e6}{\mets}?
    \item Сколько грамм сахара было растворено в 100 г воды, если ее температура замерзания опустилась до \qty{-1.3}{\celsius}?
\end{enumerate}

Фосген образуется из угарного газа и хлора в соответствии со следующим уравнением:

\begin{gather*}
    \CH{CO + Cl2 -> COCl2}
\end{gather*}

\begin{enumerate}
\setcounter{enumi}{3}
    \item Увеличение концентрации угарного газа в 2 раза приводит к увеличению начальной скорости образования фосгена в 2 раза. Определите порядок данной реакции по угарному газу.
\end{enumerate}
\problem{Название крутой задачи}{11} % Название и вес задачи
\ProblemAuthor{Авторов А.}

\ProblemPointsFour{1}{2}{3}{3}{9}{11}

\begin{solbox}{1 \ball}
    Есть два органических вещества с молекулярной формулой \CH{C2H6O} — этанол и диметиловый эфир. Из них только в первом есть гидроксо-группа, поэтому ответ — $\CH{H3C-CH2-OH}$ \em{(1 \ball)}.
\end{solbox}

\begin{solbox}{2 \balla}
    Используем формулу:

    \[ M v_{rms}^2 = 3RT \]

    Выразим $T$:

    \[ T = \frac{M v_{rms}^2}{3R} \]

    Подставим значения, и получим ответ:

    \[ T = \frac{\qty{4e-3}{\kgmol} \times (\qty{3.5e6}{\mets})^2}{3 \times \qty{8.314}{\jmolk}} = \qty{561}{\kelvin} \text{ (2 \balla)} \]
\end{solbox}

\begin{solbox}{3 \balla}
    Используем формулу, которая связывает изменение в температуре замерзания растворителя и моляльность растворенного вещества:

    \[ \Delta T_f = -i k_f m \]

    Сахар имеет формулу \CH{C12H22O11} и для него фактор Вант-Гоффа, $i$, равен единице. $k_f$ равна \qty{1.86}{\kg\celsius\per\mole} для воды.

    \[ m = \frac{\qty{-1.3}{\celsius}}{\qty{-1.86}{\kg\celsius\per\mole}} = \qty{0.70}{\mole\per\kg} \]

    Отсюда можно найти количество сахара в граммах:

    \[ m_{\text{сахар}} = \qty{0.70}{\mole\per\kg} \times \qty{0.100}{\kg} \times \qty{486}{\gmol} = \qty{34}{\g} \text{ (3 \balla)} \]
\end{solbox}

\begin{solbox}{3 \balla}
    \[ r_0 = k \cdot [\CH{CO}]_0^m \cdot [\CH{Cl2}]_0^n \]
    \[ r_1 = k \cdot [\CH{CO}]_1^m \cdot [\CH{Cl2}]_0^n \]
    \[ \frac{r_0}{r_1} = \left( \frac{[\CH{CO}]_0}{[\CH{CO}]_1} \right)^m = \left(\frac{1}{2}\right)^m = \frac12 \]
    \[ m = 1 \]

    Ответ: Порядок реакции по угарному газу равен одному (\em{3 \balla}).
\end{solbox}

\begin{solbox}{8 \ballov}
    \noindent
    \begin{tabularx}{\textwidth}{|Y|Y|Y|}
        \hline\rowcolor{gray!25}
            \textbf{I} & \textbf{II} & \textbf{III} \\
        \hline
            \chemcompound{img/structure-1} & \chemcompound{img/structure-1} & \chemcompound{img/structure-1} \\
        \hline\rowcolor{gray!25}
            \textbf{A} & \textbf{B} & \textbf{C} \\
        \hline
            \chemcompound{img/structure-2} & \chemcompound{img/structure-2} & \chemcompound{img/structure-2} \\
        \hline\rowcolor{gray!25}
            \textbf{X} & \textbf{Y} \\
        \cline{1-2}
            \chemcompound[3.5cm]{img/structure-3} & \chemcompound[3.5cm]{img/structure-3} \\
        \cline{1-2}
    \end{tabularx}
\end{solbox}
\problem{Название крутой задачи}{11}

\ProblemPointsFour{1}{2}{3}{3}{9}{11}

\answerbox{1cm}

\answerbox{2.5cm}

\answerbox{3cm}

\answerbox{4cm}

\Answerbox{2.5cm}

\problempart

\noindent
\begin{tabularx}{\textwidth}{|Y|Y|Y|}
\hline\rowcolor{gray!25}
\textbf{I} & \textbf{II} & \textbf{III} \\
\hline
\filler{2.5cm} & \filler{2.5cm} & \filler{2.5cm} \\
\hline\rowcolor{gray!25}
\textbf{A} & \textbf{B} \\
\cline{1-2}
\filler{2.5cm} & \filler{2.5cm} \\
\cline{1-2}
\end{tabularx}


\ansbox{19}{2.5cm}


\end{document}
