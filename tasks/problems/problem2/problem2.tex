\problem{В чем сила?}{11}

\ProblemPointsFour{6}{8}{4}{5}{23}{11}

Некоторые химические элементы обладают уникальными свойствами – можно только поражаться многообразию и красоте их соединений. Но, к сожалению, бывают и трудности. Например, по совершенно необъяснимой причине, некоторые химики, переболев коронавирусной инфекцией, начинают говорить либо \textbf{\underline{только} правду} (таких мы назовем \textbf{рыцарями}), либо \textbf{\underline{только} ложь} (таких мы назовем \textbf{лжецами}).

Однажды собралась компания из пяти химиков, переболевших коронавирусом. Это Антон (\textbf{А}), Богдан (\textbf{Б}), Малена (\textbf{М}), Дильназ (\textbf{Д}), Тания (\textbf{Т}). Среди них есть два рыцаря. Они обсуждали соединения элемента \textbf{X}. Если химики говорят об окислительно-восстановительных свойствах соединений, образованных из \textbf{X}, они говорят о процессах, в которых \textbf{X} изменяет свою степень окисления.

\textbf{А.:} Соединение \textbf{2} (содержит 32.84\% \textbf{X} по массе) образуется в результате реакции \textbf{X} с желто-зеленым газом \textbf{1}, состоящим из одного элемента.

\textbf{Д.:} Ну, вообще-то, газ \textbf{1}, состоящий из одного элемента, – светло-зеленый. Но да, реагируя с \textbf{X} он образует \textbf{2} (содержит 34.90\% \textbf{X} по массе).

\textbf{М.:} Ребята, вы серьезно? При комнатной температуре \textbf{1} – темно-красная жидкость (которая при испарении образует коричневый газ, но все же это жидкость), которая состоит из одного элемента и реагируя с \textbf{X} образует \textbf{2} (содержит 18.90\% \textbf{X} по массе).

\textbf{Б.:} Господа! Заметим, что \textbf{2} способно к автопротолизу. Разве это не примечательно?

\textbf{М.:} Я бы сказала примечательно то, что темно-красная жидкость \textbf{1} может вступать в реакцию с хлоридом натрия с образованием желто-зеленого газа. Магия химии, ничем не меньше!

\textbf{А.:} \textbf{X} образует оранжевый оксид \textbf{3}, массовая доля \textbf{X} в котором ровно 52.00\%!

\textbf{Д.:} Глупости! Массовая доля \textbf{X} в высшем оксиде \textbf{3} составляет 56.01\%.

\textbf{Т.:} Хотя бы давайте согласимся, что растворяясь в кислотах, высший оксид \textbf{3} образует оранжевые растворы, а растворяясь в щелочах – желтые.

\textbf{Б.:} Еще чего! Растворы \textbf{3} в кислотах --- светло-желтые, а в щелочах и вовсе бесцветные!

\textbf{Д.:} Если растворить \textbf{3} в гидроксиде натрия, получится соль \textbf{4} (содержащая 27.70\% \textbf{X} и 37.50\% натрия по массе), которая, вопреки ожиданиям, не проявляет сильных окислительных свойств. Поразительно, да?

\textbf{А.:} Ну как же так: \textbf{4} – сильный и широко применяемый окислитель.

\textbf{Т.:} Зачем вводить людей в заблуждение? Продукт растворения \textbf{3} в гидроксиде натрия, соединение \textbf{4} известный восстановитель!

\textbf{Д.:} А вы знали, что соединение \textbf{3} катализирует одну из стадий важнейшего промышленного процесса?

\textbf{М.:} Конечно, ведь соединение \textbf{3} катализирует процесс Борна-Габера.

\textbf{А.:} Элемент \textbf{X} в нулевой степени окисления образует гомолептический октаэдрический парамагнитный комплекс \textbf{5} (23.63\% \textbf{X} по массе), в состав которого входит ядовитый газ \textbf{6} с плотностью по водороду равной 14.

\textbf{Б.:} Да, плотность по водороду газа \textbf{6} равна 14, но какой же он ядовитый? Это же основной компонент воздуха!

\textbf{Д.:} Вообще-то массовая доля \textbf{X} в гомолептическом ($\ce{XL6}$) октаэдрическом комплексе \textbf{5} слегка меньше и составляет 23.25\%.

\begin{enumerate}
  \item Определите, кто в этой компании лжет, а кто – рыцарь. Приведите вашу аргументацию и покажите ваши расчеты. \textit{Подсказка:} попробуйте допустить, что человек говорит правду (или ложь) – приводит ли такое допущение к противоречиям? \textit{Подсказка:} попробуйте сначала найти всех лжецов.
  \item Определите элемент \textbf{X} и соединения \textbf{1-5}.
  \item Приведите уравнения реакций, к которым ссылались рыцари в этой компании.
\end{enumerate}

К компании присоединяется Санжар, который утверждает, что 1 моль соединения \textbf{3} содержит нечетное кол-во моль атомов, и \textbf{3} вступает в кислотно-основную реакцию с концентрированной азотной кислотой, образуя нитрат \textbf{7} (35.14\% \textbf{X} по массе). А вот с концентрированной серной кислотой, \textbf{3} вступает в окислительно-восстановительную реакцию, образуя очень красивый синий раствор соли \textbf{8} (31.25\% \textbf{X} по массе).

\begin{enumerate}
  \setcounter{enumi}{3}
  \item Кем является Санжар – лжецом или рыцарем? Если лжецом – обоснуйте, если рыцарем – укажите формулы \textbf{7} и \textbf{8}.
\end{enumerate}
