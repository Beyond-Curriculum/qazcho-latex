\problem{Химический блиц}{8}

\ProblemPointsSeven{2}{3}{3}{4}{4}{2}{3}{21}{8}

Предлагаем вам сделать небольшую интеллектуальную разминку и решить следующие задачи.

\begin{enumerate}
  \item Установите формулу оксида, в котором массовая доля кислорода равна 56.36\%.
  \item Запишите уравнения реакций разложения а) нитрата калия, б) нитрата цинка, в) нитрата серебра.
  \item Запишите уравнения реакций перманганата калия с нитритом калия в а) серной кислоте, б) воде, в) гидроксиде калия.
  \item На полное восстановление 7.57 г смеси оксидов железа (II) и меди потребовалось 2.24 л молекулярного водорода (при н. у.). Определите массовые доли оксидов в исходной смеси.
  \item ``Нужно больше олеума'' подумал химик. Какую массу 20\% (по массе) олеума необходимо добавить к 50 г 98\% (по массе) серной кислоты, чтобы получить олеум с массовой долей в 1.804\%?
  \item Запишите полную электронную конфигурацию атома меди.
  \item Определите степени оксиления каждого атома в следующих веществах: а) $\ce{K4[Fe(CN)6]}$, б) $\ce{Na2Cr2O7}$, в) $\ce{I2}$.
\end{enumerate}
