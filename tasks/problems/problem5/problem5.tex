\problem{Учимся понимать кинетику}{11}

\begin{table}[H]
    \small
    \begin{tabularx}{\textwidth}{|X|X|X|X|X|X|X|X|X|X|X|}
        \hline
        \cellcolor{gray!45} \textbf{\arabic{problemcount}.1.1} & \cellcolor{gray!45} \textbf{\arabic{problemcount}.1.2} & \cellcolor{gray!45} \textbf{\arabic{problemcount}.1.3} & \cellcolor{gray!45} \textbf{\arabic{problemcount}.1.4} & \cellcolor{gray!45} \textbf{\arabic{problemcount}.2.1} & \cellcolor{gray!45} \textbf{\arabic{problemcount}.2.2} & \cellcolor{gray!45} \textbf{\arabic{problemcount}.2.3} & \cellcolor{gray!45} \textbf{\arabic{problemcount}.2.4} & \cellcolor{gray!45} \textbf{\arabic{problemcount}.2.5} & \cellcolor{gray!45} \textbf{Всего} & \cellcolor{gray!45} \textbf{Вес(\%)} \\
        \hline
        2 & 6 & 3 & 3 & 2 & 3 & 5 & 4 & 4 & 32 & 11 \\
        \hline
    \end{tabularx}
\end{table}

\subproblem{Подзадача 1. Реакция второго порядка.}

Однозначно, многие из Вас знакомы с различными порядками реакций. Для элементарных реакций (тех, в механизме которых замешана лишь одна стадия и отсутствуют вещества отличные от самих реагентов) скорость реакции пропорциональна концентрации реагирующих веществ.

Допустим, для реакции первого порядка:

\begin{gather}
    \notag \ce{A ->[k] \text{продукты}} \\
    \label{eq:first-order-one-reag} v_1 = k \cdot [A] = - \frac{dA}{dt}
\end{gather}

\begin{enumerate}
    \item Выведите зависимость концентрации вещества А от времени, решив дифференциальное уравнение (\ref{eq:first-order-one-reag}).
\end{enumerate}

Давайте рассмотрим наипростейший пример для реакции второго порядка типа

\begin{equation*}
    \ce{A + B ->[k] \text{продукты}}
\end{equation*}

Тогда вывод кинетического уравнения будет выглядить следующим образом:

\begin{equation*}
    k \cdot [\ce{A}] \cdot [\ce{B}] = - \frac{d[\ce{A}]}{dt}
\end{equation*}

Рассмотрим случай, когда начальные концентрации $[\ce{A}]_0$ и $[\ce{B}]_0$ равны. Так как $[\ce{A}]_0 = [\ce{B}]_0$, а стехиометрические коэффициенты говорят о том, что они расходуются одинаково, мы можем сказать, что $[\ce{A}] = [\ce{B}]$ в любой момент времени.

Тогда:

\begin{gather*}
    k [\ce{A}]^2 = -\frac{d[\ce{A}]}{dt} \\
    -kdt = [\ce{A}]^{-2} d[\ce{A}] \\
    -k \int\limits_{0}^{t_1} dt = \int\limits_{[\ce{A}]_0}^{[\ce{A}]_1} [\ce{A}]^{-2} d [\ce{A}] \\
    \frac{1}{[\ce{A}]_1} = \frac{1}{[\ce{A}]_0} + kt_1
\end{gather*}

Вот мы и вывели уравнение для реакции второго порядка при равных концентрациях реагентов. А что, если концентрации не равны?

\begin{enumerate}
    \setcounter{enumi}{1}
    \item Выведите кинетическое уравнение для случая, когда $[\ce{A}]_0 \neq [\ce{B}]_0$, при этом не забывайте, что А и В расходуются с одинаковой скоростью из-за равных стехиометрических коэффициентов.

    \textit{Подсказки:} вам может пригодиться записать выражение для $[\ce{B}]$ через $[\ce{B}]_0$, $[\ce{A}]_0$, $[\ce{A}]$. Также, для решения дифференциального уравнения Вам пригодится метод неопределенных коэффициентов, при котором интеграл сложной дроби разбивается на интегралы более простых дробей. Иными словами, любую дробь слева можно разбить на сумму дробей справа:
    
    \begin{equation*}
        \frac{1}{(x-a)(x-b)} = \frac{A}{x-a} + \frac{B}{x-b}
    \end{equation*}

    \item Некоторый индикатор окрашивает раствор в красный цвет при концентрации вещества А равному 0.01\unit{\moll}. Через какое время с начала реакции раствор потеряет свою окраску, если смешали 0.50\unit{\moll} раствор А с 0.80\unit{\moll} раствором В? Константа скорости реакции 0.59\unit{\per\second}.
    
    \item Какой порядок реакции будет наблюдаться если $[\ce{A}]_0 \gg [\ce{B}]_0$ (при этом $[\ce{B}]_0 \gg 1$)?
\end{enumerate}

\subproblem{Подзадача 2. Применение квазистационарного приближения.}

Наверняка, многие из Вас знакомы с схемой ферментативного катализа «ключ-замок». Однако, это простейшая схема ферментативного катализа. В задание ниже мы рассмотрим неконкурентное ингибирование.

\begin{enumerate}
    \item В чем отличие конкурентного ингибирования от неконкурентного с точки зрения строения фермента?
\end{enumerate}

Схему неконкурентного ингибирования можно записать следующим образом:

\begin{gather*}
    \ce{E + S <=>[k_1][k_{-1}] ES ->[k_2] E + P} \\
    \ce{E + I <=>[K_I] EI}
\end{gather*}

\begin{enumerate}
    \setcounter{enumi}{1}
    \item При каком соотношении констант возможно написать квазистационарное приближение для фермент-субстратного комплекса?
    
    \item Используя квазистационарное приближение для фермент-субстратного комплекса и уравнение материального баланса для всех форм фермента, выведите кинетическое уравнение. Напомним, что $\displaystyle K_M = \frac{k_{-1} + k_2}{k_1}$
\end{enumerate}

Фермент CYP2C9 участвует в катализе окисления ксенобиотиков – веществ, чужеродных для организма. Например, чтобы пометить ненасыщенные жирные кислоты для взаимодействия с другими ферментами, CYP2C9 катализирует реакцию эпоксидирования.

\begin{enumerate}
    \setcounter{enumi}{3}
    \item Запишите продукты реакции взаимодействия линолевой кислоты (цис,цис-9,12-октадекадиеновая кислота) с CYP2C9. Продуктом не может являться соединение, содержащие сразу две эпоксидные группы.
    
    \item Ингибиторами фермента CYP2C9 являются нифедипин, транилципропин, фенилизотиоцианат, и др. Они принимаются внутрь, чтобы снизить метаболическую активность ферментов против конкретных медицинских препаратов. Сравните скорость реакции эпоксидации при концентрациях ингибитора в крови $[\ce{I}] = 0$ и $[\ce{I}] = 0.5\unit{\moll}$, если
    концентрация субстрата в конкретный момент времени 0.183\unit{\moll}, а концентрация всех форм фермента 6\unit{\milli\moll}.
    
    Данные: $k_1 = 0.0042~\unit{\lmols}; \ k_{-1} = 0.00019~\unit{\per\second}; \ k_2 = 0.089~\unit{\per\second}; \ K_I = 1.89$
\end{enumerate}