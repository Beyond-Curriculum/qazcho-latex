\problem{В чем сила? (Моргунов А.)}{11}

\ProblemPointsFour{6}{8}{4}{5}{23}{11}


\begin{solbox}{6 баллов}
  Единственная информация, которая у нас есть о соединении \em{1} исходит от Антона, Дильназ и Малены, которые, судя по всему, говорят о хлоре, фторе и броме соответственно. Допустим Малена говорит правду. Тогда \em{1} – бром. Но Малена тут же говорит о том, что $\ce{Br2}$ вступает в реакцию с $\ce{NaCl}$ c образованием $\ce{Cl2}$. Такого быть не может – вытеснять галогены из галогенидов могут только более активные галогены, а бром менее активен чем хлор. Значит Малена – лжец (1 балл).

  Допустим Тания говорит правду – тогда \em{3} – это высший оксид. Но высший оксид, растворяясь в гидроксиде натрия, никак не может быть восстановителем – элемент уже находится в высшей степени окисления. Значит Тания – лжец (1 балл).

  Поймать последнего лжеца гораздо сложнее. Допустим Антон говорит правду. Тогда и по массовым долям \em{2} и по массовым долям \em{3} становится понятно, что речь идет о хроме. В нулевой степени окисления у хрома \em{6} электронов. Газ \em{6}, являющийся ядовитым по утверждению Антона, это $\ce{CO}$. Но $\ce{CO}$ – это лиганд сильного поля, значит соединение \em{5} обязано быть низкоспиновым. Но шесть электронов в низкоспиновой конфигурации никак не могут приводить к парамагнитным свойствам. Получаем противоречие. Значит Антон – лжец (2 балла)

  Таким образом, Богдан и Дильназ рыцари (2 балла)
\end{solbox}



\begin{solbox}{8 баллов}
  Определение элемента \em{X} можно провести и по соединению \em{2}, и по соединению \em{3}.

  Соединение \em{2} фторид, с общей формулой $\ce{EF_n}$

  \begin{gather*}
    \frac{x}{x + 19n} = 0.3490 \\
    2.865 x = x + 19n \\
    1.865 x = 19 n \\
    x = 10.188 n
  \end{gather*}

  Проверим разные варианты значений $n$

  \begin{tabularx}{\textwidth}{|X|X|X|X|X|X|X|X|}
    \hline
    n & 1 & 2 & 3 & 4 & 5 & 6 & 7 \\
    \hline
    x & 10.188 & 20.376 & 30.564 & 40.752 & 50.94 & 61.128 & 71.316 \\
    \hline
  \end{tabularx}

  С одной стороны можно подумать, что \em{X} — это фосфор, а \em{2} — это $\ce{PF3}$. Однако, массовая доля фосфора в $\ce{P2O5}$ всего лишь 43.6\%. Тогда \em{X} это ванадий (1 балл), \em{1} — это $\ce{F2}$ (1 балл), а \em{2} — это $\ce{VF5}$ (1 балл)

  Подтвердить ванадий можно и с помощью расчетов по \em{3}.
  
  Определим формулу соединения \em{3}. В общем виде оксиды имеют формулу $\ce{X_nO_m}$. Возьмем атомную массу \em{X} за $x$

  \begin{gather*}
    \frac{nx}{nx + 16m} = 0.5601 \\
    1.785 nx = nx + 16m \\
    0.785 nx = 16m \\
    x = 20.382 \frac{m}{n}
  \end{gather*}
\end{solbox}
  
\begin{emptysolbox}
  Рассмотрим разные значения для $m$ и $n$
  
  \begin{tabularx}{\textwidth}{|X|X|X|X|X|X|X|X|X|}
    \hline
    n & 1 & 1 & 1 & 1 & 2 & 2 & 2 & 2 \\
    \hline
    m & 1 & 2 & 3 & 4 & 1 & 3 & 5 & 7 \\
    \hline
    x & 20.382 & 40.764 & 61.146 & 81.528 & 10.191 & 30.573 & 50.955 & 71.337 \\
    \hline
  \end{tabularx}

  Единственный подходящий вариант — комбинация $n = 2$ и $m = 5$, соответствующая элементу Ванадий (V).

  Таким образом \em{3} – $\ce{V2O5}$ (1 балл)

  Соединением \em{4} является соль натрия, которая может иметь формулу $\ce{NaVO3}$ или $\ce{Na3VO4}$ (проводя параллели с другими анионами с элементами в степени окисления +5). Проверки массовых долей подтверждают второй вариант: соединение \em{3} – это $\ce{Na3VO4}$ (1 балл)

  Наконец, основной компонент воздуха с молекулярной массой 28~\unit{\gmol} (соединение \em{6}) это азот $\ce{N2}$ (1 балл)

  Формула \em{5} необычна, но подтверждается массовыми долями: это [$\ce{V(N2)6}$] (2 балла). Справка: это соединение было выделено при 20-25~\unit{\kelvin} совместной конденсацией атомов ванадия и молекул азота.
\end{emptysolbox}



\begin{solbox}{4 балла}
  \begin{equation*}
    \ce{V + \frac{5}{2} F2 -> VF5}
  \end{equation*}

  (1 балл – образование \em{2})

  \begin{equation*}
    \ce{2VF5 <=> [VF4]+ + [VF6]-}
  \end{equation*}

  (1 балл – уравнение автопротолиза)

  Растворение оксида в гидроксиде натрия (1 балл):

  \begin{equation*}
    \ce{V2O5 + 6NaOH -> 2Na3VO4 + 3H2O}
  \end{equation*}

  Пентаоксид диванадия катализирует реакцию окисления диоксида серы (1 балл)

  \begin{equation*}
    \ce{SO2 + \frac{1}{2} O2 -> SO3}
  \end{equation*}
\end{solbox}



\begin{solbox}{3 балла}
  Санжар рыцарь (1 балл)

  Допустим в соединении \em{7} один атом ванадия. Тогда на остальные атомы приходится:

  \begin{equation*}
    \frac{50.94}{0.3514} \cdot 0.6486 = 94~\unit{\gmol}
  \end{equation*}

  Очевидно среди них есть атомы азота и кислорода, причем на каждый атом азота как минимум 3 атома кислорода. Нитратная группировка имеет массу 62~\unit{\gmol}. Остаток в 32~\unit{\gmol} может соответствовать еще двум атомам кислорода, что  соответствует форме ванадия в степени окисления +5 в кислой среде в виде диоксованадия. 

  Итого формула \em{7} - $\ce{VO2NO3}$ (2 балла). 0 баллов если в предлагаемой формуле нет катиона $\ce{VO2+}$.

  Проводя аналогичные рассуждения с \em{8}, на один атом ванадия приходится 112~\unit{\gmol} других элементов, из которых 96~\unit{\gmol} занимает сульфат. Таким образом \em{8} это сульфат ванадила:

  $\ce{VOSO4}$ (2 балла). 0 баллов если в предлагаемой формуле нет катиона $\ce{VO^2+}$
\end{solbox}
