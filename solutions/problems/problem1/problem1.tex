\problem{Химический блиц (Моргунов А.)}{8}

\ProblemPointsSeven{2}{3}{3}{4}{4}{2}{3}{21}{8}

\begin{solbox}{2 балла}
  Общая формула оксида - $\ce{E2O_y}$. Обозначим атомную массу элемента за $x$

  \begin{gather*}
    \frac{16y}{2x + 16y} = 0.5636 \\
    28.3889 y = 2x + 16y \\
    x = 6.19
  \end{gather*}

  Перебор разных значений $y$
  \vspace{1ex}

  \begin{tabularx}{\textwidth}{|X|X|X|X|X|X|X|}
    \hline
    $y=1$ & 2 & 3 & 4 & 5 & 6 & 7 \\
    \hline
    $x=6.19$ & 12.39 & 18.58 & 24.77 & 30.97 & 37.16 & 43.36 \\
    \hline
  \end{tabularx}

  \vspace{1ex}
  Значит это фосфор, а формула оксида $\ce{P2O5}$
\end{solbox}


\begin{solbox}{3 балла}
  За каждую реакцию по 1 баллу (0 баллов без коэффициентов)

  \begin{gather*}
    \ce{2KNO3 -> 2KNO2 + O2} \\
    \ce{2Zn(NO3)2 -> 2ZnO + 4NO2 + O2} \\
    \ce{2AgNO3 -> 2Ag + 2NO2 + O2}
  \end{gather*}
\end{solbox}


\begin{solbox}{3 балла}
  За каждую реакцию по 1 баллу (0 баллов без коэффициентов)

  \begin{gather*}
  \ce{2KMnO4 + 5KNO2 + 3H2SO4 -> 2MnSO4 + 5KNO3 + K2SO4 + 3H2O} \\
  \ce{2KMnO4 + 3KNO2 + H2O -> 2MnO2 + 3KNO2 + 2KOH} \\
  \ce{2KMnO4 + KNO2 + 2KOH -> 2K2MnO4 + KNO3 + H2O}
  \end{gather*}

  По 0.5 б если реакции неправильные, но верно указан продукт восстановления перманганата калия
\end{solbox}


\begin{solbox}{7 баллов}
  Запишем уравнения реакций:

  \begin{gather*}
    \ce{FeO + H2 -> Fe + H2O} \\
    \ce{CuO + H2 -> Cu + H2O}
  \end{gather*}

  \qty{2.24}{\liter} водорода при н.у. это

  \[
    \frac{\qty{2.24}{\liter}}{\qty{22.4}{\liter\per\mole}} = \qty{0.1}{\mole}
  \]

  Допустим в смеси было $x$ моль $\ce{FeO}$ и $y$ моль $CuO$. Тогда:

  \begin{gather*}
    71.85x + 79.55y = 7.57 \\
    x + y = 0.1
  \end{gather*}

  Решая систему получаем $x = 0.05, \ y = 0.05$

  Тогда:

  \begin{gather*}
    \omega(\ce{FeO}) = \frac{0.05 \cdot 71.85}{7.57} \cdot 100\% = 47.46\% \\
    \omega(\ce{CuO}) = \frac{0.05 \cdot 79.55}{7.57} \cdot 100\% = 52.54\%
  \end{gather*}
\end{solbox}

\begin{emptysolbox}
    По 2 балла за каждую массовую долю, всего 4 балла. Если ученик использовал целые атомные массы и получил ответ 51.12\% оксида железа и 48.88\% оксида меди – по 1.5 балла за каждую массовую долю.
\end{emptysolbox}




\begin{solbox}{4 балла}
  При добавлении олеума протекает следующая реакция с водой из серной кислоты: (1 балл за реакцию)

  \[
    \ce{H2O + SO3 -> H2SO4}
  \]

  При этом, поскольку в итоге образуется олеум, вся вода должна прореагировать: (1 балл за эту идею)

  \[
    \nu(\ce{H2O}) = \frac{0.02 \cdot 50}{18} = \frac{1}{18}
  \]

  Таким образом, если добавить $x$ грамм 20\% олеума, после смешения останется:

  \[
    m(\ce{SO3}) = 0.2 x - \frac{80.06}{18}
  \]

  А общая масса раствора будет $50+x$ Таким образом:

  \[
    \frac{0.2 x - \frac{80.06}{18}}{50 + x} = 0.01804
  \]

  Отсюда $x=\qty{29.38}{\gram}$ (2 балла)
\end{solbox}




\begin{solbox}{2 балла}
  2 балла за такую конфигурацию

  \[
    1s^2 2s^2 2p^6 3s^2 3p^6 4s^1 3d^{10}
  \]

  1 балл если указана

  \[
    1s^2 2s^2 2p^6 3s^2 3p^6 4s^2 3d^9
  \]
\end{solbox}



\begin{solbox}{3 балла}
  \begin{enumerate}[label=\Alph*)]
    \item Калий +1, Железо +2, углерод +2, азот -3 (за каждую по 0.25 б, всего 1 б)
    \item Натрий +1, хром +6, кислород -2 (за натрий и кислород по 0.25 б, за хром 0.5 б, всего 1 б)
    \item Йод – 0 (1 балл)
  \end{enumerate}
\end{solbox}
